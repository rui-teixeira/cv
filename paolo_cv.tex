%
% Plasmati Graduate CV
% LaTeX Template
% Version 1.0 (24/3/13), original
% Version 1.1 (4/6/13), modified by Fernando Paolo
%
% V1.0 of this template has been downloaded from:
% http://www.LaTeXTemplates.com
%
% History:
% Original by Alessandro Plasmati (alessandro.plasmati@gmail.com)
% Modified by Fernando Paolo (fspaolo@gmail.com)
%
% License:
% CC BY-NC-SA 3.0 (http://creativecommons.org/licenses/by-nc-sa/3.0/)
%
% Important notes:
% This template needs to be compiled with XeLaTeX.
% The main document font is called Fontin and can be downloaded for 
% free from here: http://www.exljbris.com/fontin.html.
% 
% To intall the fonts on Mac OS X open the `Font Book` and drag 
% the folder `Fontin` to it.
%
% Appears to be a problem with the *default* output driver `xdvipdfmx` 
% and the Fontin fonts. So to compile use istead `xdv2pdf`:
%
%   xelatex -output-driver=xdv2pdf thisfile.tex
%
% See http://www.tug.org/pipermail/xetex/2008-January/008163.html
%

%-------------------------------------------------------------------------------
%	PACKAGES AND OTHER DOCUMENT CONFIGURATIONS
%-------------------------------------------------------------------------------

% Default font size and paper size
\documentclass[a4paper,11pt]{article} 

% For loading fonts
\usepackage{fontspec} 
\defaultfontfeatures{Mapping=tex-text}
% Main document font
\setmainfont[SmallCapsFont=Fontin SmallCaps,
             ItalicFont=Fontin Italic,
             BoldFont=Fontin Bold]{Fontin} 

% Formatting packages
\usepackage{xunicode,xltxtra,url,parskip} 

% Required for specifying custom colors
\usepackage[usenames,dvipsnames]{xcolor} 

% Margin formatting of the A4 page
%\usepackage[big]{layaureo} 
% An alternative to layaureo can be 
\usepackage{fullpage}
% To reduce the height of the top margin uncomment: 
%\addtolength{\voffset}{-1.3cm}

% Required for adding links and customizing them
\usepackage{hyperref} 
% Link color
\definecolor{linkcolour}{rgb}{0,0.2,0.6} 
% Set link colors throughout the document
\hypersetup{colorlinks,breaklinks,urlcolor=linkcolour,linkcolor=linkcolour} 

% Used to customize the \section command
\usepackage{titlesec} 
% Text formatting of sections
\titleformat{\section}{\Large\scshape\raggedright}{}{0em}{}[\titlerule] 
% Spacing around sections
\titlespacing{\section}{0pt}{3pt}{3pt} 

\begin{document}

% Removes page numbering
\pagestyle{empty} 

% Change the font of the \LaTeX command under the skills section
%\font\fb=''[cmr10]'' 

%-------------------------------------------------------------------------------
%	NAME AND CONTACT INFORMATION
%-------------------------------------------------------------------------------

\par{\centering{\Huge Fernando S. \textsc{Paolo}}\bigskip\par} % Your name

\section{Contact Info}

\begin{tabular}{rl}
\textsc{Institution:} & University of California, San Diego \\
\textsc{Address:} & 9500 Gilman Dr, La Jolla, 92093 \\
\textsc{Phone:} & +1 (858) 534-9643\\
\textsc{Email:} & \href{mailto:fpaolo@ucsd.edu}{fpaolo@ucsd.edu} \\
\textsc{Website:} & \href{http://fspaolo.net}{fspaolo.net}
\end{tabular}

%-------------------------------------------------------------------------------
%	EDUCATION
%-------------------------------------------------------------------------------

\section{Education}

\begin{tabular}{l}
Ph.D. \textsc{Earth Sciences}, \textbf{Scripps Institution of Oceanography}, Univ. of California, SD (\emph{Ongoing})\\
M.S. \textsc{Geophysics} (w/honors), University of S\~ao Paulo, Brazil 2009\\
B.S. \textsc{Oceanography} (w/honors), University of S\~ao Paulo, Brazil 2007\\
Tech. \textsc{Graphic Arts}, Municipal School of Tech. Education, Argentina 1996\\
\end{tabular}

%-------------------------------------------------------------------------------
%	RESEARCH INTERESTS
%-------------------------------------------------------------------------------

\section{Research Interests}

\begin{tabular}{p{.95\textwidth}}
Satellite remote sensing of ice sheets, ice-ocean-atmosphere interaction, ice-shelf/sheet mass balance, climate change | Radar and laser altimetry, radar interferometry and imagery, processing and analysis of big data.
\end{tabular}

%-------------------------------------------------------------------------------
%	WORK EXPERIENCE 
%-------------------------------------------------------------------------------

\section{Research Experience}

\begin{tabular}{r|p{12.4cm}}
\emph{Current} & \textsc{Doctoral Dissertation} at \textsc{ucsd} \\
2010--\textsc{Present} & \small{Interannual and decadal variations of Antarctic ice shelves using multi-mission satellite radar altimetry, and links with oceanic and atmospheric forcings.}\\
\multicolumn{2}{c}{} \\

%------------------------------------------------

2007--2009 & \textsc{Master's Thesis} at \textsc{usp}\\
& \small{Satellite altimetry and marine gravity on the integrated representation of the gravity field in the Brazilian coast.}\\
\multicolumn{2}{c}{} \\

%------------------------------------------------

2004--2006 & \textsc{Bachelor's Thesis} at \textsc{usp}\\
& \small{Characterization of bottom morphodynamics of Canan\'eia estuary by shallow seismic profiling and side scan sonar.}\\
\multicolumn{2}{c}{} \\

%------------------------------------------------

2003--2004 & \textsc{Undergraduate Research} at \textsc{usp}\\
& \small{Study on hydrocarbon concentrations in water and sediment from King George Island, Antarctica (\emph{including field work}).}
\end{tabular}

%-------------------------------------------------------------------------------
%	FIELD EXPERIENCE
%-------------------------------------------------------------------------------

\section{Field Experience}

\begin{tabular}{rl}
\textsc{Dec} 2004 & RV \emph{P.W. Besnard}, Geophysical and Geological Survey, Brazil coast \footnotesize{(12 days at sea)}\\
\textsc{Jan} 2004 & Brazilian Antarctic Research Station \emph{Comandante Ferraz} \footnotesize{(30 days in Antarctica)}\\
\textsc{Dec} 2003 & RB \emph{Escuna}, Environmental Monitoring Program, Bahia, Brazil \footnotesize{(14 days at sea)}\\
\textsc{Jan} 2003 & RV \emph{P.W. Besnard}, Oceanographic Moorings II, Brazil coast \footnotesize{(8 days at sea)}\\
\textsc{Jul} 2002 & RV \emph{P.W. Besnard}, Oceanographic Moorings I, Brazil coast \footnotesize{(9 days at sea)}
\end{tabular}

%-------------------------------------------------------------------------------
%	TEACHING EXPERIENCE
%-------------------------------------------------------------------------------

\section{Teaching Experience}

\begin{tabular}{rl}
\textsc{Aug--Dec} 2008 & Computing for Geophysicists, TA at University of S\~ao Paulo\\
\textsc{Mar--Jul} 2008 & Introduction to Geophysics, TA at University of S\~ao Paulo
\end{tabular}

%-------------------------------------------------------------------------------
%	SKILLS 
%-------------------------------------------------------------------------------

\section{Some Skills}

\begin{tabular}{rl}
\textsc{Programming:} & Python, C/C++, Fortran 90/95, Shell-Script, \textsc{html}/\textsc{css}/JavaScript\\
\textsc{Softwares:} & \textsc{unix}/Mac/\textsc{Linux}, \textsc{vtk}/VisIt/Paraview, \textsc{gmt}, \textsc{Matlab}, \textsc{gis}, \LaTeX, \textsc{vim}, etc.\\
\textsc{Languages:} & English (fluent), Portuguese (native), Spanish (native)
\end{tabular}

%-------------------------------------------------------------------------------
%	AWARDS AND HONORS
%-------------------------------------------------------------------------------

\section{Awards and Honors}

\begin{tabular}{rl}
2012, 2013 & \textsc{nasa} Earth and Space Science Fellowship (\textsc{nessf})\\ 
2011 & \textsc{aaaspd} Student Award (1\textsuperscript{st} place), Atmospheric and Oceanographic Sciences\\
2010 & \textsc{agu} Outstanding Student Paper Award, Cryosphere\\
2010 & Honor Mention (best M.S. Thesis in geophysics), University of S\~ao Paulo\\
2008 & TA Fellowship (M.S.), Brazilian Ministry of Education\\
2007, 2008 & Graduate Fellowship (M.S.), Brazilian Ministry of Science and Technology\\
2007 & Honor Mention (2\textsuperscript{nd} best B.S. Thesis), University of S\~ao Paulo\\
2005, 2006 & Undergrad. Fellowship, S\~ao Paulo Research Foundation\\
2004 & Undergrad. Fellowship, Brazilian Ministry of Science and Technology
\end{tabular}

%-------------------------------------------------------------------------------
%	PUBLICATIONS
%-------------------------------------------------------------------------------

\section{Publications}

%\subsection*{Journal Articles}
\emph{Journal Articles}

%\begin{small}
\begin{enumerate}
  \item [3.] {\bf Paolo, F.S.} and E.C. Molina (2010), Integrated marine 
        gravity field in the Brazilian coast from altimeter-derived sea 
        surface gradient and shipborne gravity. {\it J. Geodyn.}, 50, 347-354,\\
        doi:10.1016/j.jog.2010.04.003.
  \item [2.] B\'icego, M.C., E. Zanardi-Lamardo, S. Taniguchi, C.C. Martins, 
        D.A.M. da Silva, S.T. Sasaki, A.C.R. Albergaria-Barbosa, {\bf F.S. 
        Paolo}, R.R. Weber and R.C. Montone (2009), Results from a 15-year 
        study on hydrocarbon concentrations in water and sediment from 
        Admiralty Bay, King George Island, Antarctica. {\it Antarct. Sci.}, 
        21, 209-220, doi:10.1017/S0954102009001734.
  \item [1.] {\bf Paolo, F.S.} and M.M. Mahiques (2008), Utilization of 
        acoustic methods in coastal dynamics studies: example in the 
        Canan\'eia lagoonal mouth. {\it Braz. J. Geophys.}, 26(2), 211-225, 
        doi:10.1590/S0102-261X2008000200008.
\end{enumerate}
%\end{small}

%-------------------------------------------------------------------------------
%	INVITED TALKS
%-------------------------------------------------------------------------------

\section{Invited Talks}

\begin{tabular}{rl}
14 \textsc{Jun} 2011 & Elevation changes on Antarctic ice shelves, \textsc{aaas} Pacific Division, 92\textsuperscript{nd} Annual\\ 
& Meeting, San Diego.
\end{tabular}

%-------------------------------------------------------------------------------
%	INTERESTS AND ACTIVITIES
%-------------------------------------------------------------------------------

%\section{Other Interests and Activities}
%
%Technology, Open-Source, Programming, Design, Communication\\

%-------------------------------------------------------------------------------
%       FOOTER
%-------------------------------------------------------------------------------

\bigskip
\begin{center}
  \begin{footnotesize}
  \textcolor[gray]{0.5}{Last updated: \today}\\
  \end{footnotesize}
\end{center}

\newpage

%-------------------------------------------------------------------------------

\end{document}
