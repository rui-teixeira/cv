%
% Version
% -------
% Version 1.2 (3/1/15), Fernando Paolo
% Version 1.1 (4/6/13), Fernando Paolo
% Version 1.0 (24/3/13), Alessandro Plasmati
%
% Modified from 'Plasmati Graduate CV' by Fernando Paolo <fspaolo@gmail.com>
%
% V1.0 of this template has been downloaded from:
% http://www.LaTeXTemplates.com
%
% License
% -------
% CC BY-NC-SA 3.0 (http://creativecommons.org/licenses/by-nc-sa/3.0/)
%
% Notes
% -----
% This template needs to be compiled with XeLaTeX.
% The main document font is called Fontin and can be downloaded for 
% free from here: http://www.exljbris.com/fontin.html.
% 
% To intall the fonts on Mac OS X open the 'Font Book' and drag 
% the folder 'Fontin' to it.
%
% Appears to be a problem with the *default* output driver 'xdvipdfmx' 
% and the Fontin fonts. So to compile the document use istead 'xdv2pdf':
%
%   xelatex -output-driver=xdv2pdf paolo_cv.tex
%
% See http://www.tug.org/pipermail/xetex/2008-January/008163.html
%
% Alternatively, to compile using the standard 'pdflatex' with the Times 
% fonts, comment the lines marked with 'xdv2pdf', and uncomment the ones 
% with 'pdflatex'. Then just run:
%
%   pdflatex paolo_cv.tex
%
%-------------------------------------------------------------------------------
%	PACKAGES AND OTHER DOCUMENT CONFIGURATIONS
%-------------------------------------------------------------------------------

% Default font size and paper size
\documentclass[a4paper,11pt]{article} 

% left-align text
\usepackage[document]{ragged2e}

% For loading fonts
\usepackage{fontspec}                         % xdv2pdf
\defaultfontfeatures{Mapping=tex-text}        % xdv2pdf

% Main document font
\setmainfont[SmallCapsFont=Fontin SmallCaps,  % xdv2pdf
             ItalicFont=Fontin Italic,        % xdv2pdf
             BoldFont=Fontin Bold]{Fontin}    % xdv2pdf

% Alternative fonts if Fontin is not available
%\usepackage{times}                           % pdflatex

% Formatting packages
\usepackage{xunicode,xltxtra,url,parskip}     % xdv2pdf

% Required for specifying custom colors
\usepackage[usenames,dvipsnames]{xcolor} 

% Margin formatting of the A4 page
\usepackage[bottom=1cm, top=2.5cm, right=2.5cm, left=2cm]{geometry}

%\usepackage[big]{layaureo} 
% An alternative to layaureo can be 
%\usepackage{fullpage}
% To reduce the height of the top margin uncomment: 
%\addtolength{\voffset}{-1.3cm}

% Used to customize the \section command
\usepackage{titlesec} 

% Text formatting of sections (uncomment for title rule)
%\titleformat{\section}{\Large\scshape\raggedright}{}{0em}{}[\titlerule] 
\titleformat{\section}{\Large\scshape\raggedright}{}{0em}{}

% Spacing around sections
\titlespacing{\section}{0pt}{3pt}{3pt} 

% Subheader color and symbols
\definecolor{grey}{rgb}{0.35,0.35,0.35}
\newcommand*{\tbullet}{~\strut\textcolor{grey}{\LARGE\textbullet}~~}
\newcommand*{\tdash}{\strut\textcolor{grey}{\textemdash}~}

% Microtype
%\usepackage[
%activate={true,nocompatibility},
%final,
%tracking=true,
%kerning=true,
%spacing=true,
%factor=1100,
%stretch=10,
%shrink=10
%]{microtype}
% activate={true,nocompatibility} - activate protrusion and expansion
% final - enable microtype; use "draft" to disable
% tracking=true, kerning=true, spacing=true - activate these techniques
% factor=1100 - add 10% to the protrusion amount (default is 1000)
% stretch=10, shrink=10 - reduce stretchability/shrinkability (default is 20/20)

% Required for adding links and customizing them
\usepackage{hyperref} 
% Link color
\definecolor{linkcolour}{rgb}{0,0.2,0.6} 
% Set link colors throughout the document
\hypersetup{colorlinks,breaklinks,urlcolor=linkcolour,linkcolor=linkcolour} 

\begin{document}

% Removes page numbering
\pagestyle{empty} 

% Change the font of the \LaTeX command under the skills section
%\font\fb=''[cmr10]'' 

%-------------------------------------------------------------------------------
%	NAME AND CONTACT INFORMATION
%-------------------------------------------------------------------------------

\begin{center}
{\Huge 
Fernando S. \textsc{Paolo}
}\\[0.2cm]
%
\textcolor{grey}{
739 Paularino Ave. Apt. A103 \tdash 
Costa Mesa CA 92626 \tdash
USA
}\\[0.01cm]
%
\href{}{+1~(858)~752~1106} \tbullet 
\href{mailto:fpaolo@ucsd.edu}{fpaolo@ucsd.edu} \tbullet 
\href{http://fspaolo.net}{www.fspaolo.net}
\end{center}
\vspace*{0.7cm}

%\section{Contact Info}
%
%\begin{tabular}{rl}
%\textsc{Institution:} & University of California, San Diego \\
%\textsc{Address:} & 9500 Gilman Dr, La Jolla, 92093 \\
%\textsc{Phone:} & +1~(858)~534~9643\\
%\textsc{Email:} & \href{mailto:fpaolo@ucsd.edu}{fpaolo@ucsd.edu} \\
%\textsc{Website:} & \href{http://fspaolo.net}{fspaolo.net}
%\end{tabular}

%-------------------------------------------------------------------------------
%	EDUCATION
%-------------------------------------------------------------------------------

\section{Education}

\begin{tabular}{l}
Ph.D. Geophysics, \textbf{Scripps Oceanography}, University
    of California, San Diego (\emph{Sep 2, 2015})\\
M.S. Geophysics (w/honors), University of S\~ao Paulo, Brazil 2009\\
B.S. Oceanography (w/honors), University of S\~ao Paulo, Brazil 2007\\
%Tech. Graphic Arts, Muni. School of Tech. Education, Argentina 1996\\
\end{tabular}

%-------------------------------------------------------------------------------
%	RESEARCH INTERESTS
%-------------------------------------------------------------------------------

\section{Interests}

\begin{tabular}{p{\textwidth}}
Satellite remote sensing, ice-ocean-atmosphere interaction, ice-shelf/ice-sheet mass balance, climate change ~|~ Radar and laser altimetry, radar interferometry and imagery ~|~ Processing and analysis of large-scale data sets, image processing, statistical modeling and time-series analysis
\end{tabular}

%-------------------------------------------------------------------------------
%	WORK EXPERIENCE 
%-------------------------------------------------------------------------------

\section{Research}

\begin{tabular}{r|p{.85\textwidth}}
\emph{Current} & \textsc{Doctoral Dissertation} \\
2009--15 & {Interannual and decadal variations of Antarctic ice shelves using multi-mission satellite radar altimetry, and links with oceanic and atmospheric forcings.}\\
\multicolumn{2}{c}{} \\  % for blank line

%------------------------------------------------

2007--08 & \textsc{Master's Thesis} \\
& {Satellite altimetry and marine gravity on the integrated representation of the gravity field along the Brazilian coast.}\\
\multicolumn{2}{c}{} \\

%------------------------------------------------

2004--06 & \textsc{Bachelor's Thesis} \\
& {Characterization of bottom morphodynamics of Canan\'eia estuary by shallow seismic profiling and side-scan sonar.} \\
%\multicolumn{2}{c}{}

%------------------------------------------------

%2003--04 & \textsc{Undergraduate Research} \\
%& {Study on hydrocarbon concentrations in water and sediment from King George Island, Antarctica (\emph{including fieldwork}).}
\end{tabular}

%-------------------------------------------------------------------------------
%	SKILLS 
%-------------------------------------------------------------------------------

\section{Skills}

\begin{tabular}{rp{.78\textwidth}}
\textsc{Programming:} & Python, C/C++, Fortran 90/95, Shell-Script, HTML/CSS/JavaScript\\
\textsc{Software:} & UNIX/Mac/Linux, VTK/ParaView, GMT, Matlab, MPI, GIS, \LaTeX, git, VIM, etc.\\
\textsc{Languages:} & English (fluent), Portuguese (native), Spanish (native)
\end{tabular}

%-------------------------------------------------------------------------------
%	COMMUNICATION
%-------------------------------------------------------------------------------

\section{Communication}

\begin{tabular}{p{\textwidth}}
Received training in Public Speaking \& Leadership (at Toastmasters and UC San Diego) ~|~ Gave interviews (about my research) to {\it Los Angeles Times}, {\it The Washington Post}, {\it Reuters}, {\it BBC News}, {\it The Wall Street Journal}, among others ~|~ Wrote 6 papers, 3 articles and presented at 12 (inter)national conferences (3 awarded prizes) ~|~ Co-wrote lab grants and proposals ~|~ Reviewed papers for {\it J. Glaciol.} and {\it The Cryosphere} ~|~ Assisted in reviewing NASA and NSF proposals ~|~ Organized 2 institutional seminar series ~|~ Developed and maintain 3 websites
\end{tabular}

%-------------------------------------------------------------------------------
%	TEACHING EXPERIENCE
%-------------------------------------------------------------------------------

\section{Teaching}

\begin{tabular}{rl}
\textsc{Aug--Dec} 2008 & Computing for Geophysicists, T.A. at University of S\~ao Paulo\\
\textsc{Mar--Jul} 2008 & Introduction to Geophysics, T.A. at University of S\~ao Paulo
\end{tabular}

%-------------------------------------------------------------------------------
%	AWARDS AND HONORS
%-------------------------------------------------------------------------------

\section{Awards and Honors}

\begin{tabular}{rl}
2014 & AGU Outstanding Student Paper Award, Cryosphere\\
2013--15 & NASA Earth and Space Science Fellowship (NESSF)\\
2011 & AAAS Student Award (1st place), Atmospheric \& Oceanographic Sci.\\
2010 & AGU Outstanding Student Paper Award, Cryosphere\\
2010 & Honor Mention (best M.S. Thesis in geophysics), University of S\~ao Paulo\\
2008 & Brazilian Ministry of Education Teaching Fellowship\\
2007--08 & Brazilian Ministry of Sci. \& Tech. Fellowship (Masters)\\
2007 & Honor Mention (2nd best B.S. Thesis), University of S\~ao Paulo\\
2005--06 & S\~ao Paulo Research Foundation Fellowship (Undergrad)\\
2004 & Brazilian Ministry of Sci. \& Tech. Fellowship (Undergrad)
\end{tabular}

%-------------------------------------------------------------------------------
%	PUBLICATIONS
%-------------------------------------------------------------------------------

\section{Publications}

%\subsection*{Journal Articles}
\emph{Journal Articles}

\begin{minipage}{\textwidth}
\begin{flushleft}
\begin{itemize}
  \item[---] {\bf F.~S.~Paolo}, H.~A.~Fricker, L.~Padman, ``Developing optimal
        decadal records of Antarctic ice-shelf height change from multiple
        satellite radar altimeters'', \emph{Remote. Sens. Environ.} (in review).
  \item[---] P.~R.~ Holland, A.~Brisbourne, H.~F.~J.~Corr, D.~McGrath, K.~Purdon, 
        J.~Paden, H.~A.~Fricker, {\bf F.~S.~Paolo}, A.~Fleming, ``Oceanic and 
        atmospheric forcing of Larsen C Ice-Shelf thinning'', \emph{The Cryosphere} 
        (2015).
        %doi:10.5194/tc-9-1005-2015
  \item[---] {\bf F.~S.~Paolo}, H.~A.~Fricker, L.~Padman, ``Volume loss 
        from Antarctic ice shelves is accelerating'', \emph{Science} (2015).
        %doi:10.1126/science.aaa0940
  \item[---] {\bf F.~S.~Paolo}, E.~C.~Molina, ``Integrated marine 
        gravity field along the Brazilian coast from altimeter-derived sea 
        surface gradient and shipborne gravity'', \emph{J. Geodyn.} (2010).
        %doi:10.1016/j.jog.2010.04.003
  \item[---] {\bf F.~S.~Paolo}, ``Satellite altimetry and marine gravity on the
        integrated representation of the gravity field, Brazil coast'',
        {\it M.S. Thesis} (2009).
        %doi.org/10.1590/S0102-261X2010000300014
  \item[---] M.~C.~B\'icego, E.~Zanardi-Lamardo, S.~Taniguchi, C.~C.~Martins, 
        D.~A.~M.~da~Silva, S.~T.~Sasaki, A.~C.~R.~Albergaria-Barbosa, {\bf F.~S.~Paolo},
        R.~R.~Weber, R.~C.~Montone, ``Results from a 15-year 
        study on hydrocarbon concentrations in water and sediment from 
        Admiralty Bay, King George Island, Antarctica'', \emph{Antarct. Sci.} 
        (2009).
        %doi:10.1017/S0954102009001734
  \item[---] {\bf F.~S.~Paolo}, M.~M.~Mahiques, ``Utilization of 
        acoustic methods in coastal dynamics studies: example in the 
        Canan\'eia lagoonal mouth'', \emph{Braz. J. Geophys.} (2008).
        %doi:10.1590/S0102-261X2008000200008
\end{itemize}
\end{flushleft}
\end{minipage}

\emph{Data Sets}

\begin{minipage}{0.98\textwidth}
\begin{itemize}
  \item [---] Moholdt, G., H.A. Fricker, L. Padman and {\bf F.S. Paolo} (2013), 
        Synthesized grounding line and ice shelf mask for Antarctica.
        {\it Scripps Institution of Oceanography, UCSD}, 25 pp.\\
        doi.pangaea.de/10.1594/PANGAEA.819150
\end{itemize}
\end{minipage}

%-------------------------------------------------------------------------------
%	FIELD EXPERIENCE
%-------------------------------------------------------------------------------

\section{Fieldwork}

\begin{tabular}{rl}
\textsc{Dec} 2004 & RV \emph{P.W. Besnard}, Geophysical and Geological Survey, Brazil coast \footnotesize{(12 days at sea)}\\
\textsc{Jan} 2004 & Brazilian Antarctic Research Station \emph{Comandante Ferraz} \footnotesize{(30 days in Antarctica)}\\
\textsc{Dec} 2003 & RB \emph{Escuna}, Environmental Monitoring Program, Bahia, Brazil \footnotesize{(14 days at sea)}\\
\textsc{Jan} 2003 & RV \emph{P.W. Besnard}, Oceanographic Moorings II, Brazil coast \footnotesize{(8 days at sea)}\\
\textsc{Jul} 2002 & RV \emph{P.W. Besnard}, Oceanographic Moorings I, Brazil coast \footnotesize{(9 days at sea)}
\end{tabular}

%-------------------------------------------------------------------------------
%	INVITED TALKS
%-------------------------------------------------------------------------------

%\section{Invited Talks}
%
%\begin{tabular}{rl}
%14 \textsc{Jun} 2011 & Elevation changes on Antarctic ice shelves, \textsc{aaas} Pacific Division, 92\textsuperscript{nd} Annual\\ 
%& Meeting, San Diego.
%\end{tabular}

%-------------------------------------------------------------------------------
%	ACADEMIC SERVICE
%-------------------------------------------------------------------------------

%\section{Academic Service}
%
%\begin{tabular}{rl}
%& Reviewer for: Journal of Glaciology ~|~ The Cryosphere Discussion\\
%Date & Organized the: Polar Seminar Series ~|~ Earth Section Seminar Series\\
%Date & Website development \& maintenance: glaciology2.ucsd.edu/igs2011 ~|~ glaciology.ucsd.edu\\
%Date & Student representative undergrad
%\end{tabular}

%-------------------------------------------------------------------------------
%	MEMBERSHIPS
%-------------------------------------------------------------------------------

%\section{Memberships}
%
%\begin{tabular}{l}
%American Geophysical Union ~|~ Toastmasters International
%\end{tabular}

%-------------------------------------------------------------------------------
%	INTERESTS AND ACTIVITIES
%-------------------------------------------------------------------------------

%\section{Other Interests and Activities}
%
%Technology, Open-Source, Design, Communication, Philosophy\\

%-------------------------------------------------------------------------------
%	RFERENCES
%-------------------------------------------------------------------------------

\section{References}

\begin{small}
\begin{tabular}{p{.98\textwidth}}
Dr. Helen A. Fricker, {\it Remote Sensisng and Glaciology}, UC San Diego, hafricker@ucsd.edu\\
Dr. Laurie Padman, {\it Physical Oceanography}, Earth \& Space Research, padman@esr.org\\
Dr. David T. Sandwell, {\it Remote Sensing and Geodesy}, UC San Diego, dsandwell@ucsd.edu
\end{tabular}
\end{small}

%-------------------------------------------------------------------------------
%       FOOTER
%-------------------------------------------------------------------------------

\bigskip
\begin{center}
  \begin{footnotesize}
  \textcolor[gray]{0.45}{Last updated: \today}\\
  \end{footnotesize}
\end{center}

\newpage

%-------------------------------------------------------------------------------

\end{document}

%- public speaking workshop
%- professional communication workshop
%- public speaking and leadership training
%- cyber infrastructure and big data workshop
%- some international talks (not necessarily invited, as the one in the CV)
%- managing seminar series
%- website development and maintenance
%- Karthaus, Remote Sensing Reading
%- Summer School Brazil?

